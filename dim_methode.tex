% % % % % % % % % % % % % % % % % % % %% % % % % % % % % %% % % % % % % % % %
% Fiche méthode adimensionnement 
% Version: 15 juillet 2021
% % % % % % % % % % % % % % % % % % % %% % % % % % % % % %% % % % % % % % % %
\documentclass[12pt]{book}
\usepackage{teach}
\usepackage[utf8]{inputenc}
\usepackage[T1]{fontenc}
\usepackage[french]{babel}
\usepackage{amsmath,amssymb}
\usepackage{framed}
\usepackage{enumitem}
\usepackage{palatino}
\usepackage{natbib}
\usepackage[sectionbib]{bibunits}
\usepackage{pxfonts}
\usepackage{hyperref}
\usepackage[squaren,Gray,cdot]{SIunits}
\usepackage{xcolor}
\usepackage{graphicx}
\usepackage{caption}

\hypersetup{
	colorlinks=True,
	urlcolor=\colorlink,
	linkcolor=\colorlink,
	citecolor=\colorlink
           }
\newcommand{\todo}[1]{{\color{red} [TODO: #1]}}
\renewcommand{\FrenchLabelItem}{\textbullet}
% ----------------------------------------------------------------------------
% Notations
% ----------------------------------------------------------------------------
\newcommand{\mitbf}[1]{\hbox{\mathversion{bold}$#1$}}
\renewcommand{\phi}{\varphi}

%Constante
\newcommand{\pieps}{\dfrac{1}{4\pi\epsilon_0}}

%Vecteurs
\newcommand{\vece}{\mitbf{E}}
\newcommand{\vecg}{\mitbf{g}}
\newcommand{\vecn}{\mitbf{n}}
\newcommand{\vecd}{\mathrm{\textbf{d}}}
\newcommand{\vecm}{\mitbf{m}}
\newcommand{\vecM}{\mitbf{M}}
\newcommand{\vecs}{\mitbf{S}}
\newcommand{\veca}{\mitbf{A}}
\newcommand{\vecH}{\mitbf{H}}
\newcommand{\vecb}{\mitbf{B}}
\newcommand{\vecf}{\mitbf{F}}
\newcommand{\vecj}{\mitbf{j}}
\newcommand{\vecv}{\mitbf{v}}
\newcommand{\er}{\mitbf{e_r}}
\newcommand{\etheta}{\mitbf{e_\theta}}
\newcommand{\ephi}{\mitbf{e_\phi}}
\newcommand{\ex}{\mitbf{e_x}}
\newcommand{\ey}{\mitbf{e_y}}
\newcommand{\ez}{\mitbf{e_z}}
\newcommand{\ds}{\mathrm{d}\mitbf{S}}
\renewcommand{\vec}[1]{\mathbf{#1}}
\newcommand{\complex}[1]{\underline{#1}}

%Derivative
\newcommand{\grad}{\mitbf{\nabla}}
\newcommand{\laplacien}{\grad^2}
\newcommand{\dV}{\mathrm{d}V}
\newcommand{\dtheta}{\mathrm{d}\theta}
\newcommand{\dphi}{\mathrm{d}\varphi}
\newcommand{\dr}{\mathrm{d}r}
\newcommand{\dt}{\mathrm{d}t}
\newcommand{\dx}{\mathrm{d}x}
\newcommand{\dy}{\mathrm{d}y}
\newcommand{\dz}{\mathrm{d}z}
\newcommand{\dl}{\mathrm{\textbf{d}}\mitbf{\ell}}
\renewcommand{\div}{\mathrm{div}\,}
\newcommand{\rot}{\mathrm{\textbf{rot}}\,}
\newcommand{\gradient}{\mathrm{\textbf{grad}}\,}
\newcommand{\ddtheta}{\partial \theta}
\newcommand{\ddphi}{\partial \varphi}
\newcommand{\ddr}{\partial r}
\newcommand{\ddx}{\partial x}
\newcommand{\dd}[2]{\dfrac{\partial #1}{\partial #2}}
\newcommand{\dn}[2]{\dfrac{\mathrm{d} #1}{\mathrm{d} #2}}

% ----------------------------------------------------------------------------
% Documents
% ----------------------------------------------------------------------------
\begin{document}
\def\author{Théo Tassin}
\def\title{L2 STEP - Électromagnétisme}
\parskip 10pt plus2pt minus2pt
\chapter*{Analyse dimensionnelle}
\section*{Objectifs}
\begin{itemize}
	\item Déterminer la dimension d'une grandeur
	\item Vérifier l'homogénéité d'une expression
\end{itemize}

\begin{defn}[Analyse dimensionnelle]
	\begin{itemize}
		\item L'analyse dimensionnelle est une méthode 
		  pratique permettant de 
		  vérifier \emph{l'homogénéité} d'une formule physique en décomposant les 
		  grandeurs physiques qu'elle met en jeu en un produit de grandeurs 
		  de base : le système d'unité international 
		  (d'après \textit{Wikipédia}).
		 \item Deux grandeurs sont homogènes si elles s'expriment dans les
		   mêmes unités.
	\end{itemize}
\end{defn}

\section{Le système d'unité international et les unitées dérivées}
Le système international est le système d'unités le plus couramment utilisé dans 
le monde. Il comporte 7 unités de base résumées dans le Tableau~\ref{usi}.

\begin{table}[ht!]
	\centering
	\caption{Unités de base du système international (SI)}
	\begin{tabular}{lll}
	\textbf{Grandeur} & \textbf{Unités} & \textbf{Symbole} \\ \hline
	Masse & kilogramme & $\kilogram$ \\ 
	temps & seconde & $\second$ \\
	distance & mètre & $\meter$ \\
	température & kelvin & $\kelvin$ \\
	intensité électrique & ampère & $\ampere$ \\
	quantité de matière & mole & $\mole$ \\
	intensité lumineuse & candela & $\candela$
	\end{tabular}
	\label{usi}
\end{table}

De ces unités de base, on peut dériver d'autres unités. On peut ainsi former une
infinité d'unités dérivées. Le Tableau~\ref{usi_deriv} fournit quelques exemples.

\begin{table}[ht!]
	\centering
	\caption{Unités dérivées du système international}
	\begin{tabular}{lll}
		\textbf{Grandeur} & \textbf{Unité dérivée} & \textbf{Décomposition}\\ \hline
		Énergie & joule $\joule$ & $\kilogram \usk \meter \squared \usk \second \rpsquared$\\
		Vitesse &  $\meter \usk \rp \second$ & $\meter \usk \rp \second$\\
		Champ magnétique & tesla $\tesla$ & $\kilogram \usk \rp \ampere \usk \second \rpsquared$ \\ 
		Force & newton $\newton$  & $\kilogram \usk \meter \usk \second \rpsquared$ \\
		Charge & coulomb $\coulomb$ &  $\ampere \usk \second$
	\end{tabular}
	\label{usi_deriv}
\end{table}

\begin{defn}[Unité dérivée]
Une \emph{unité dérivée} est une unité qui combine plusieurs unités fondamentales
(sous la forme d'un produit de puissances de plusieurs de ces unités de base).
Ces unités dérivées ont parfois leur propre nom et symbole comme le newton
$\newton$. Toute unité dérivée s'exprime en fonction des unités de base du système 
international. (d'après \textit{Wikipédia})

\end{defn}


\section{Vérifier l'homogénéité d'une expression}
Pourquoi s'embêter à vérifier l'homogénéité d'une expression~?
La réponse est simple : cela permet \emph{d'éviter des erreurs grossières~!} En effet,
si j'aboutis à l'expression $a = b$, $a$ doit avoir la même dimension que $b$. 
Si ce n'est pas le cas, je sais qu'il faut revoir mes calculs et/ou mon raisonnement.
En plus d'éviter des erreurs, la dimension d'un objet physique 
peut permettre de mieux comprendre à quoi il correspond. 

\begin{attention}
	L'homogénéité d'une expression n'assure pas sa validité~!
\end{attention}

La vérification de l'homogénéité d'une expression peut être compliquée par la présence de certaines opérations,
telle que la dérivation, ou par la présence d'unités dérivés. Nous allons maintenant voir comment surmonter 
ces difficultés.
\subsection{Multiplication, dérivation et intégration}
Quelques règles simples à retenir

\begin{enumerate}
	\item Si on multiplie deux grandeurs, l'unité du résultat est 
	  la multiplication des unités des grandeurs de départ. Par exemple,
	  une surface ($\meter \squared$)
	  est homogène  à une distance au carré car elle correspond
	  au produit d'une distance par une distance.
	\item Dériver une grandeur par une autre revient, en terme de dimension,
		à diviser l'unité de la première par l'unité de la deuxième. Une vitesse ($\meter \usk \rp \second$)
		est obtenue en dérivant une distance ($\meter$) par un temps ($\second$) 
		et est donc homogène à une distance
	  divisée par un temps. Pensez à la notation 
	  $\dfrac{\mathrm{d}x}{\mathrm{d}t}$ pour vous en souvenir : dériver
	  $x$ par rapport à $t$ revient grossièrement à diviser une petite variation de $x$
	  par une petite variation de $t$ (Ne faites pas ça pour dériver une fonction~!).
	\item Inversement, lorsqu'on intègre une grandeur par rapport à une autre, cela
	  revient à multiplier l'unité de la première par l'unité de la deuxième.
	  Une pression $(\newton \usk \meter \rpsquared$) intégrée sur une surface ($\meter \squared$) 
	  est donc homogène à une force ($\newton$).
\end{enumerate}

\begin{rema}
\begin{itemize}
	\item Le gradient, la divergence et le rotationnel correspondent à des 
	opérations de dérivation.
	\item On ne peut pas additionner deux grandeurs de dimensions 
	  différentes~!
\end{itemize}
\end{rema}
\subsection{Les unitées dérivées}
Les unités dérivées qui possèdent leur propre nom peuvent poser problème car 
on a tendance à oublier leur décomposition en unités SI.
Comment s'en sortir dans ce cas~? Les deux exemples
suivants illustrent chacun une manière de s’en sortir dans certaines
situations. 

\subsubsection{Trouver une expression simple de cette unité dérivée}
Imaginons que j'ai oublié comment les joules se décomposent dans le système international.
Il suffit alors de me rappeler que l'énergie cinétique $E_c$ est une énergie et s'exprime
comme le produit d'une masse $m$ par une vitesse $v$ au carré

\begin{equation*}
	E_c = \dfrac{1}{2} m v^2.
\end{equation*}
J'en conclus que 
$\joule = \kilogram \usk \meter \squared \usk \second \rpsquared$. Cette méthode 
peut s'appliquer à d'autres exemples
\begin{itemize}
	\item Joule : énergie cinétique (produit d'une masse par une vitesse au carré),
	\item Newton : poids (produit d'une masse par une accélération),
	\item Pascal : pression hydrostatique (produit d'une masse volumique par une accélération et par une distance).
\end{itemize}

\subsubsection{Se ramener à une expression connue}
Un fil rectiligne infini uniformément chargé avec une densité linéïque de charge
$\lambda$ ($\coulomb \usk \rp \meter$) créé en un point $M$ de l'espace
situé à un point $r$ du fil, un champ électrique $E$

\begin{equation*}
	E = \dfrac{\lambda}{4 \pi \epsilon_0 r},
\end{equation*}
où $\epsilon_0$ est la permittivité diélectrique du vide en 
$\farad \usk \rp \meter$. Comment vérifier l'homogénéité de cette expression
lorsqu'on a oublié la décomposition des Farad en unité SI~? On essaye de se 
ramener à une expression connue du champ électrique, celui généré par une charge 
$Q$ à une distance $r$

\begin{equation*}
	E_Q = \dfrac{Q}{4 \pi \epsilon_0 r^2}.
\end{equation*}
D'après cette formule, un champ électrique est homogène à une charge divisée 
par $\epsilon_0$ et une distance au carré. Pour essayer de retrouver une expression
semblable, on multiplie le numérateur et le dénominateur de $E$ par $r$

\begin{equation*}
	E = \dfrac{\lambda r}{4 \pi \epsilon_0 r^2}.
\end{equation*}
$\lambda r$ étant homogène à une charge, on conclut que cette expression est 
bien homogène.

\end{document}
