\documentclass[9pt, dvipsnames,aspectratio=169]{beamer}

\usepackage[utf8]{inputenc} 		% Output encoding
\usepackage[frenchb]{babel} % Multi language support
\usepackage{csquotes}			  % Multi language quotes
\usepackage{caption}
\usepackage{hyperref}
\usepackage{amssymb}
% Math symbols and font
%\usepackage{amsmath, amssymb}
%\usefonttheme{professionalfonts}
%\usepackage[euler-hat-accent]{eulervm}
\usepackage{bm}
\usepackage{color}
\usepackage{multicol}
\usepackage{doi}
\usepackage[]{hyperref}
\usepackage[squaren,Gray,cdot]{SIunits}
%Bibliography
\usepackage[style=authoryear,backend=biber,maxcitenames=2]{biblatex}
\renewcommand*{\nameyeardelim}{\addcomma\addspace}
\addbibresource{Presentation.bib}

\usetheme{metropolis} 			% Beamer theme
\usepackage{graphicx} 						% Import graphics
%\usepackage{hyperref}	% hypertext in doc, % metadata
\usepackage{xcolor}	% nicer colors for hyperref links
\usepackage{appendixnumberbeamer} % Additional slides not counted
%\usepackage{siunitx}
%\usepackage{multimedia}						% Videos
\usepackage{luatex85}
\usepackage{hhline}
\usepackage{xcolor}

% Metropolis options
\setbeamercolor{background canvas}{bg=white}
\setbeamerfont{caption}{size=\tiny}
\metroset{progressbar=frametitle}
\metroset{numbering=fraction}

% Hyperlinks
\hypersetup{
	colorlinks  = true, % Colours links instead of ugly boxes
	unicode 	= true, % Unicode characters in metadata
	urlcolor    = BlueViolet, % Colour for external hyperlinks
	linkcolor   = RoyalBlue, % Colour of internal links
        citecolor   = RoyalBlue % Colour of citations
}

%Define color
\definecolor{neworange}{rgb}{0.932,0.464,0}
\definecolor{newblue}{rgb}{0.228,0.372,0.804}

\newcommand\Author{Licence 2 - Sciences de la Terre parcours Terre - 
Environnement}
\newcommand\Title{Électromagnétisme}
\newcommand\Laboratory{Université de Paris \\ Institut de Physique du Globe de Paris}
\newcommand\Group{Géomagnétisme}
\title{\Title}	% Beamer specific
\begin{document}

\begin{frame}
	\begin{flushleft}
		\Huge{\textcolor{newblue}{\textbf{\Title}}} 
	\end{flushleft}
		\vspace{-0.7cm}
		\textcolor{neworange}{\rule{\linewidth}{1pt}}\\
	\begin{center}
		{\large{\Author}}\\
		\large{Encadrants : Théo Tassin$^1$ et 
		Alexandre Fournier}
	\end{center}
	\begin{flushleft}
		\textsc{\large{\Laboratory}}\\
	\end{flushleft}
	\begin{center}
		\begin{minipage}{0.5\linewidth}
			\centerline{\includegraphics[width=0.5\linewidth]
			{logo.pdf}}
		\end{minipage}

	\end{center}
	$^1$\texttt{tassin@ipgp.fr}
\end{frame}


\begin{frame}
	\frametitle{Pré-requis}
	\begin{alertblock}{Mécanique du point}
	\begin{itemize}
		\item Être capable de résoudre un problème
		 de dynamique simple
	\end{itemize}
	\end{alertblock}
	\vfill
	\begin{alertblock}{Électrostatique}
	\begin{itemize}
		\item Connaître la loi de Coulomb
		\item Connaître la relation entre champ électrostatique
		  et potentiel électrostatique
		\item Savoir comment calculer le champ électrique
		 issue d'une distribution simple de charges
	\end{itemize}
	\end{alertblock}
	\vfill
	\begin{alertblock}{Outils mathématiques}
	\begin{itemize}
		\item Dérivation d'une fonction à 
		 plusieurs variables
	        \item Intégration sur une surface ou un contour
		\item Repère sphérique
	\end{itemize}
	\end{alertblock}
\end{frame}

\begin{frame}
	\frametitle{Compétences à valider}
	\vfill
	\begin{itemize}
		\item[\checkmark] Connaître les 
		  \textcolor{neworange}{équations de Maxwell} 
		  en régime permanent et en régime variable
		\item[\checkmark] Savoir interpréter physiquement les 
		 équations de Maxwell
		\item[\checkmark] Être capable de calculer le champ 
		  \textcolor{neworange}{électrostatique} 
		  résultant d'une distribution simple de charges
	        \item[\checkmark] Être capable de calculer le champ 
		  \textcolor{neworange}{magnétostatique}
		  résultant d'une distribution simple de courants
		\item[\checkmark] Savoir modéliser le comportement du 
		  \textcolor{neworange}{champ électrique dans un conducteur}
		\item[\checkmark] Être capable de décrire le comportement de 
		  \textcolor{neworange}{matériaux para- et ferromagnétiques} 
		  soumis à un champ magnétique
	\end{itemize}
\end{frame}

\begin{frame}
	\frametitle{Objectifs secondaires du cours}
	\begin{alertblock}{Ce que j'aimerais en plus vous faire passer comme 
			  message durant ce cours} 
	\begin{enumerate}
		\item L'utilité de l'électromagnétisme en 
		  sciences de la Terre
	        \item L'importance de vérifier l'homogénéité d'une 
		  expression
		\item L'importance de discuter physiquement une équation et 
		  un résultat numérique
	\end{enumerate}
        \end{alertblock}
\end{frame}

\begin{frame}
	\frametitle{Organisation du cours}
	\begin{alertblock}{Organisation générale}
	\begin{itemize}
		\item $16 \hour$ de cours 
		\item $16 \hour$ de Travaux dirigés : Théo Tassin et Alexandre Fournier
		\item Tous les documents du cours peuvent être téléchargés sur 
		  \href{https://moodle.u-paris.fr/}{Moodle}
		\item Utilisation ponctuelle de Wooclap en cours
	\end{itemize}
	\end{alertblock}
	\vfill
	\begin{alertblock}{Plan}
	\begin{enumerate}
		\item Champ électrique dans un conducteur
		\item Magnétostatique
		\item Étude macroscopique de l'aimantation
		\item Électromagnétisme en régime variable
	\end{enumerate}
	\end{alertblock}
	\vfill
	\begin{alertblock}{Évaluation}
	\begin{itemize}
		\item Deux devoirs maison ($40\,\%$)
		\item Un partiel final ($60\,\%$)
	\end{itemize}
	\end{alertblock}
\end{frame}

\begin{frame}[standout]
	Des questions~?
\end{frame}
\end{document}

