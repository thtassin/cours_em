\section{Exercices}

\begin{exocor}[Émission radioactive] 
	Une masse radioactive, ponctuelle, initialement neutre, située au point $O$,
	émet, à partir de l'instant $t = 0$, des particules $\alpha$ avec une vitesse
	$v_0$ supposée constante et de façon isotrope. À l'instant $t$, la charge 
	électrique située en $O$ est
	\begin{equation*}
		q(t) = q_0 \left[\exp\left(-\dfrac{t}{\tau}\right) - 1\right],
	\end{equation*}
	où $\tau$ correspond au temps de demi-vie de la masse radioactive. 
	On se place dans un repère sphérique $(O, \er, \etheta, \ephi)$. Un point
	$M$ de l'espace est repéré par ses coordonnées $(r, \theta, \phi)$. 
	On admet que le champ magnétique est nul en tout point de 
	l'espace, ce qui se démontre par des arguments de symétrie. 
	\begin{enumerate}
		\item Déterminer le champ électrique $\vece(M, t)$ 
		  en un point $M$ de l'espace pour $t > 0$.
		  Commenter.
		\item Exprimer la densité volumique de charge $\rho(M, t)$ et la
		  densité volumique de courant $\vecj(M, t)$ pour $t > 0$.
		\item Vérifier la compatibilité des résultats obtenus avec la 
		  relation locale de conservation de la charge et les équations de 
		  Maxwell.
	\end{enumerate}
\end{exocor}

\begin{exocor}[OPPH dans le vide illimité]
	\begin{enumerate}
		\item Établir l'équation de propagation du champ électrique dans 
		  le vide.
		\item Les directions de l'espace sont indiquées par la base 
		  orthonormée $(\ex, \ey, \ez)$. On envisage une solution sous 
		  forme d'onde plane progressive harmonique polarisée rectilignement
		  \begin{equation*}
			  \vece(z, t) = E_0 \cos(\omega t - k z)\ex,
		  \end{equation*}
		  où $E_0$ est l'amplitude de l'onde, $\omega$ sa pulsation temporelle
		  et $k > 0$ la pulsation spatiale. Dans quelle direction se propage
		  l'onde électromagnétique ? Quelle est l'état de polarisation 
		  de l'onde ?
	  	\item Quelle relation doivent vérifier $k$ et $\omega$ ? Utiliser
		  l'équation de propagation de $\vece$ pour aboutir à cette relation.
	  \item Déterminer le champ magnétique $\vecb(z, t)$ associé à cette onde.
	  \item Exprimer le vecteur de Poynting $\vec{\Pi}(z, t)$. 
		  En déduire la puissance 
		  $\mathcal{P}$ (moyennée en temps) traversant une surface d'aire
		  $S$ orthogonale à la direction de propagation et orientée
		  dans le sens de la propagation.
	  \item Exprimer la densité volumique $u_\mathrm{em}(z, t)$ d'énergie 
		  électromagnétique de l'onde. Que dire des termes électrique
		  et magnétique ? Moyenner $u_\mathrm{em}$ en temps. 
		\item Exprimer de deux manières différentes l'énergie qui passe 
		  à travers $S$ durant une durée $\dt$. En déduire
		  la vitesse $v_e$ de propagation de l'énergie.
	\end{enumerate}
\end{exocor}

\begin{exocor}[Onde électromagnétique plane progressive]
	On étudie une onde électromagnétique dans un repère cartésien 
	$(O, \ex, \ey, \ez)$ dont le champ électrique 
	s'exprime en notation complexe
	\begin{equation*}
		\complex{\vece}(x, y, z, t) = \complex{E_x}(x, y, z, t) \ex + 
		\complex{E_y}(x, y, z, t)\ey 
		\quad \mathrm{avec} \quad
		\complex{E_x}(x, y, z, t) = E_0 \exp\left\{i
			\left[\dfrac{k}{3}(2x + 2y + z) - \omega t
		\right]\right\},
	\end{equation*}
	avec $\omega$ la pulsation temporelle de l'onde et $k$ une constante.
	L'onde se propage dans le vide et sa longueur d'onde $\lambda$ 
	vaut $\lambda = \unit{700}{\nano \meter}$.
	\begin{enumerate}
		\item Calculer la fréquence de l'onde. À quel domaine du spectre 
		  électromagnétique cette onde appartient-elle ?
		\item Calculer la valeur numérique de la constante $k$.
		\item Exprimer $\complex{E_y}$ en fonction de $\complex{E_x}$.
		\item Calculer le champ magnétique $\vecb(x, y, z, t)$ 
		  associée à cette onde.
		\item Calculer la densité moyenne d'énergie électromagnétique associée
		  à cette onde ainsi que sa moyenne temporelle.
	  \item Calculer le vecteur de Poynting $\vec{\Pi}(x, y, z, t)$ 
		  de cette onde et sa moyenne
		  temporelle. Commenter.
	\end{enumerate}
\end{exocor}

\begin{exocor}[Onde dans un métal]
	À suffisamment basse fréquence, un métal est localement neutre et sa
	conductivité $\gamma$ est réelle. On peut y négliger le courant de 
	déplacement devant le courant de conduction.
	\begin{enumerate}
		\item Établir l'équation de propagation vérifiée par le champ
		 électrique dans le métal.
		\item Le métal est illimité dans l'espace. On envisage une 
		  onde dont le champ électrique s'écrit, en notation complexe, 
		  \begin{equation}
			  \complex{\vece}(z, t) = E_0 \exp[i(\omega t - \complex{k}z)]\ex,
		 \end{equation}
		 où $E_0$ est une constante réelle positive. Établir la relation
		 de dispersion en faisant intervenir une distance caractéristique 
		 $\delta$ (épaisseur de peau). Donner l'expression du champ 
		 électrique. Quelle est la signification de $\delta$ ?
	 \item Établir l'expression du champ magnétique $\complex{\vecb}$ de l'onde. 
		\item Établir l'expression du vecteur de Poynting moyenné en temps.
		\item On raisonne sur un parallélépipède d'épaisseur
		  $\dz$, d'extension $L$ selon $x$ et $\ell$ selon $y$. Déterminer
		  l'expression de la puissance $\langle \mathcal{P} \rangle$ 
		  (moyennée en temps)
		  cédée à ce volume de métal par l'onde (effet Joule).
		\item En moyenne, l'énergie contenue dans ce volume reste constante.
		  En réalisant un bilan énergétique sur le volume, vérifier 
		  la cohérence des résultats des deux questions précédentes.
	\end{enumerate}
\end{exocor}

\newpage


