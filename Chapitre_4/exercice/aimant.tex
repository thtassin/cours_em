\section{Exercices}

\begin{exocor}[Estimation de la taille du noyau] 
	En première approximation, le champ géomagnétique peut-être assimilé
	à un champ magnétique dipolaire. On imagine alors que le champ à la surface
	résulte de la présence d'un aimant placé au centre de la Terre.
	\begin{enumerate}
		\item On note $m$ la norme du moment magnétique de l'aimantant terrestre. 
		L'amplitude $B$ du champ magnétique à la surface de la Terre 
		vaut alors approximativement
		\begin{equation*}
			B = \dfrac{\mu_0 m}{4 \pi R_T^3},
		\end{equation*}
		où $R_T$ est le rayon de la Terre. Estimer $m$.
	\item En déduire le nombre $N$ d'atomes de la matière aimantée constituant le noyau
	  terrestre. On considère pour cela que chaque atome porte un moment
	  magnétique $\mu_B \approx \unit{10^{-23}}{\ampere \usk \meter \squared}$ 
	  de l'ordre du magnéton de Bohr.
	\item On considère que le noyau terrestre est constitué d'un mélange de fer
	  et de nickel de masse molaire $M = \unit{57}{\gram \usk \rp \mole}$
	  et de masse volumique $\rho = \unit{8}{\kilogram \usk \rp \liter}$.
	  Estimer le volume $V$ occupé par la matière aimantée.
	\item En déduire le rayon $R$ du noyau de la Terre supposé sphérique.
	  Commenter le résultat et discuter le modèle. 
	\item La température du noyau étant de l'ordre de $\unit{4000}{\celsius}$,
	  de quelle grosse lacune souffre notre modèle ?
	  
	\end{enumerate}
\end{exocor}

\begin{exocor}[Composante horizontale du champ géomagnétique]
	Un instrument, destiné à mesurer la composante horizontale $B_h$
	du champ magnétique
	terrestre, est constitué d'une boussole dont l'aiguille horizontale est 
	placée entre deux bobines de Helmholtz de rayon $R = \unit{5}{\centi \meter}$. 
	Lorsque les bobines ne sont pas alimentées, l'aiguille de la boussole 
	est orthogonale à leur axe de révolution.
	En revanche, lorsque ces enroulements
	sont parcourus par un courant $I$ l'aiguille dévie d'un angle $\phi$
	qu'un curseur orthogonal à l'aiguille permet de mesurer sur le cadran.
	\begin{enumerate}
	  \item Expliquer pourquoi l'aiguille dévie d'un angle $\phi$.
    	  \item Quelle est la condition d'équilibre de la boussole ? 
	    En déduire une expression de $\tan \phi$ en fonction de $I$.
	    On rappelle qu'au centre de deux bobines de Helmholtz l'amplitude du champ 
	    magnétique vaut $B_0 = \mu_0 I/2R$ (voir TD précédent).
	  \item En mai 2009, à Toulouse, la mesure a donné $\phi = \unit{30}{\degres}$
	    pour $I = \unit{1.08}{\ampere}$. En déduire la valeur de
	    la composante horizontale du champ magnétique à Toulouse. 
	\end{enumerate}
\end{exocor}

\begin{exocor}[Susceptibilité et loi de Curie]
	L'objectif de cet exercice est de retrouver la loi de Curie à l'aide
	d'un modèle microscopique simple. Cette loi décrit
	la dépendance de la susceptibilité magnétique $\chi$ d'un matériau à la température
	$T$
	\begin{equation*}
		\chi = \dfrac{C}{T},
	\end{equation*}
	où $C$ est la constante de Curie.

	On considère un cristal paramagnétique de volume $V$ constitué d'un ensemble
	de $N$~moments magnétiques atomiques $\vecm$ identiques et indépendants. 

	Le solide est plongé dans un champ magnétique $\vecb = B \ez$. La projection
	du moment magnétique de chaque atome est quantifiée et peut prendre deux
	valeurs suivant l'axe $(Oz)$: $m_\pm = \pm \mu_B$ où $\mu_B$ est le magnéton de
	Bohr.
	\begin{enumerate}
		\item Donner l'expression de l'énergie magnétique $E_+$ d'un moment
		  magnétique $m_+$ composant le cristal. 	
	  	\item On suppose que le cristal est maintenu à une température
		  $T$ à l'aide d'un thermostat. Les moments magnétiques 
		  n'intéragissant pas entre eux, la physique statistique
		  nous donne le nombre $N_\pm$ de moments magnétiques $m_\pm$
		  \begin{equation*}
			  N_\pm = \dfrac{1}{Z} \exp\left(\dfrac{-E_\pm}{k_B T}\right),
		  \end{equation*}
		  où $E_+$ est l'énergie d'un moment magnétique $m_\pm$,
		  $Z$ un facteur de normalisation et 
		  $k_B = 1.23 \times 10^{-23}$\, SI est la constante de Boltzmann.
		  Cette loi est appelée la loi de Boltzmann, elle est 
		  extrêmement utile en physique.
		  Déterminer les dimensions de $Z$ et $k_B$.
		
	  \item Déterminer $N_+$ et $N_-$ en explicitant l'expression de 
	    $Z$, de $E_+$ et $E_-$. En déduire le moment magnétique $M$ 
	    du cristal.
	  \item Calculer la valeur du champ $B_M$ à partir de laquelle l'énergie 
	    par moment magnétique est négligeable devant l'agitation thermique.
	    Cette valeur vous semble-t-elle accessible expérimentalement ?
	    Conclure.
	  \item L'aimantation $M_c$ du cristal s'écrit $M/V$. En déduire l'expression
	    de la susceptibilité $\chi$ de ce dernier.
	    On rappelle pour cela le développement
	    limité de $\tanh$ près de $0$
	    \begin{equation*}
		    \tanh(x) = x -\dfrac{x^3}{3} + \dfrac{2x^5}{15} - 
		    \dfrac{17x^7}{315}+ o(x^7).
	    \end{equation*}
	    En déduire l'expression de $C$.

	  \item Réaliser l'application numérique pour un cristal de cérium qui contient
	    une mole de moments dipolaires atomiques. le cérium présente une masse molaire
	    $\mathcal{m} = \unit{140}{\gram \usk \rp \mole}$ et une masse volumique
    $\rho = \unit{6.7 \times 10^3}{\kilogram \usk \rpcubic \meter}$.
	\end{enumerate}
\end{exocor}
\newpage

