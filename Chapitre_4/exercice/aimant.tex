\section{Exercices}

\begin{exocor}[Estimation de la taille du noyau] 
	En première approximation, le champ géomagnétique peut-être assimilé
	à un champ magnétique dipolaire. On imagine alors que le champ à la surface
	résulte de la présence d'un aimant placé au centre de la Terre.
	\begin{enumerate}
		\item On note $m$ la norme du moment magnétique de l'aimantant terrestre. 
		L'amplitude $B$ du champ magnétique à la surface de la Terre 
		vaut alors approximativement
		\begin{equation*}
			B = \dfrac{\mu_0 m}{4 \pi R_T^3},
		\end{equation*}
		où $R_T$ est le rayon de la Terre. Estimer $m$.
	\item En déduire le nombre $N$ d'atomes de la matière aimantée constituant le noyau
	  terrestre. On considère pour cela que chaque atome porte un moment
	  magnétique $\mu_B \approx \unit{10^{-23}}{\ampere \usk \meter \squared}$ 
	  de l'ordre du magnéton de Bohr.
	\item On considère que le noyau terrestre est constitué d'un mélange de fer
	  et de nickel de masse molaire $M = \unit{57}{\gram \usk \rp \mole}$
	  et de masse volumique $\rho = \unit{8}{\kilogram \usk \rp \liter}$.
	  Estimer le volume $V$ occupé par la matière aimantée.
	\item En déduire le rayon $R$ du noyau de la Terre supposé sphérique.
	  Commenter le résultat et discuter le modèle. 
	\item La température du noyau étant de l'ordre de $\unit{4000}{\celsius}$,
	  de quelle grosse lacune souffre notre modèle ?
	  
	\end{enumerate}
\end{exocor}

\begin{exocor}[Composante horizontale du champ géomagnétique]
	Un instrument, destiné à mesurer la composante horizontale $B_h$
	du champ magnétique
	terrestre, est constitué d'une boussole dont l'aiguille horizontale est 
	placée entre deux bobines de Helmholtz de rayon $R = \unit{5}{\centi \meter}$. 
	Lorsque les bobines ne sont pas alimentées, l'aiguille de la boussole 
	est orthogonale à leur axe de révolution.
	En revanche, lorsque ces enroulements
	sont parcourus par un courant $I$ l'aiguille dévie d'un angle $\phi$
	qu'un curseur orthogonal à l'aiguille permet de mesurer sur le cadran.
	\begin{enumerate}
	  \item Expliquer pourquoi l'aiguille dévie d'un angle $\phi$.
    	  \item Quelle est la condition d'équilibre de la boussole ? 
	    En déduire une expression de $\tan \phi$ en fonction de $I$.
	    On rappelle qu'au centre de deux bobines de Helmholtz l'amplitude du champ 
	    magnétique vaut $B_0 = \mu_0 I/2R$ (voir TD précédent).
	  \item En mai 2009, à Toulouse, la mesure a donné $\phi = \unit{30}{\degres}$
	    pour $I = \unit{1.08}{\ampere}$. En déduire la valeur de
	    la composante horizontale du champ magnétique à Toulouse. 
	\end{enumerate}
\end{exocor}

\begin{exocor}[Susceptibilité et loi de Curie]
	L'objectif de cet exercice est de retrouver la loi de Curie à l'aide
	d'un modèle microscopique simple. Cette loi décrit
	la dépendance de la susceptibilité magnétique $\chi$ d'un matériau à la température
	$T$
	\begin{equation*}
		\chi = \dfrac{C}{T},
	\end{equation*}
	où $C$ est la constante de Curie.

	On considère un cristal paramagnétique de volume $V$ constitué d'un ensemble
	de $N$~moments magnétiques atomiques $\vecm$ identiques et indépendants. 

	Le solide est plongé dans un champ magnétique $\vecb = B \ez$. La projection
	du moment magnétique de chaque atome est quantifiée et peut prendre deux
	valeurs suivant l'axe $(Oz)$: $m_\pm = \pm \mu_B$ où $\mu_B$ est le magnéton de
	Bohr.
	\begin{enumerate}
		\item Donner l'expression de l'énergie magnétique $E_+$ d'un moment
		  magnétique $m_+$ composant le cristal. 	
	  	\item On suppose que le cristal est maintenu à une température
		  $T$ à l'aide d'un thermostat. Les moments magnétiques 
		  n'intéragissant pas entre eux, la physique statistique
		  nous donne le nombre $N_\pm$ de moments magnétiques $m_\pm$
		  \begin{equation*}
			  N_\pm = \dfrac{1}{Z} \exp\left(\dfrac{-E_\pm}{k_B T}\right),
		  \end{equation*}
		  où $E_+$ est l'énergie d'un moment magnétique $m_\pm$,
		  $Z$ un facteur de normalisation et 
		  $k_B = 1.23 \times 10^{-23}$\, SI est la constante de Boltzmann.
		  Cette loi est appelée la loi de Boltzmann, elle est 
		  extrêmement utile en physique.
		  Déterminer les dimensions de $Z$ et $k_B$.
		
	  \item Déterminer $N_+$ et $N_-$ en explicitant l'expression de 
	    $Z$, de $E_+$ et $E_-$. En déduire le moment magnétique $M$ 
	    du cristal.
	  \item Calculer la valeur du champ $B_M$ à partir de laquelle l'énergie 
	    par moment magnétique est négligeable devant l'agitation thermique.
	    Cette valeur vous semble-t-elle accessible expérimentalement ?
	    Conclure.
	  \item L'aimantation $M_c$ du cristal s'écrit $M/V$. En déduire l'expression
	    de la susceptibilité $\chi$ de ce dernier.
	    On rappelle pour cela le développement
	    limité de $\tanh$ près de $0$
	    \begin{equation*}
		    \tanh(x) = x -\dfrac{x^3}{3} + \dfrac{2x^5}{15} - 
		    \dfrac{17x^7}{315}+ o(x^7).
	    \end{equation*}
	    En déduire l'expression de $C$.

	  \item Réaliser l'application numérique pour un cristal de cérium qui contient
	    une mole de moments dipolaires atomiques. Le cérium présente une masse molaire
	    $\mathcal{M} = \unit{140}{\gram \usk \rp \mole}$ et une masse volumique
    $\rho = \unit{6.7 \times 10^3}{\kilogram \usk \rpcubic \meter}$.
	\end{enumerate}
\end{exocor}
\newpage
\section{Corrigé}
\begin{corrige}
\begin{enumerate}
	\item À la surface, le champ magnétique terrestre est de l'ordre de
	  $\unit{50}{\micro \tesla}$. Le rayon de la Terre vaut approximativement
	  $\unit{6400}{\kilo \meter}$. On a donc
	  
	  \begin{equation*}
		  \boxed{ m = \dfrac{4 \pi B R_T^3}{\mu_0} \approx \unit{10^{23}}{\ampere
		  \usk \meter \squared}.} 
	  \end{equation*}

	 \item On fait ici l'hypothèse que tous les moments magnétiques atomiques
	   sont alignés et orientés dans le même sens. On a alors
	   \begin{equation*}
		   m = N\mu_B \iff \boxed{N = \dfrac{m}{\mu_B} = 10^{46}.}
	   \end{equation*}

	 \item Le noyau contient $N$ atomes. Pour obtenir le nombre de mole 
	   $n$ que cela représente, il suffit de diviser $N$ par le nombre d'
	   Avogadro $\mathcal{N}_A = \unit{6.022 \times 10^{23}}{\rp \mole}$.
	   On a alors $n = N/\mathcal{N}_A$. On peut alors remonter à la masse 
	   $m_N$ du noyau en utilisant sa masse molaire $m_N = nM$. Finalement,
	   le volume s'écrit
	   \begin{equation*}
		   V = \dfrac{m_N}{\rho} = \boxed{\dfrac{mM}{\mu_B \mathcal{N}_A \rho}
	   \approx \unit{1.2 \times 10^{17}}{\cubic \meter}.}
	   \end{equation*}

   \item On considère que le noyau de la Terre est sphérique. Son rayon $R_N$ 
     s'écrit donc
     \begin{equation*}
	     R_N = \left(\dfrac{3V}{4\pi}\right)^{1/3} \approx 
	     \boxed{\unit{300}{\kilo \meter}.}
     \end{equation*}
     En réalité le noyau interne a un rayon de $\unit{1200}{\kilo \meter}$. Nous 
     avons donc sous-estimé ce dernier. En effet, notre principal erreur a été 
     de considérer que les moments magnétiques atomiques étaient alignés
     alors que la température dans le noyau est très élevée !
  \item La température du noyau est supérieur à la température du fer qui est alors
    paramagnétique. Il n'y a donc aucune raison que les moments dipolaires atomiques
    produisent un champ macroscopique. Le champ magnétique produit par la
    Terre résulte de phénomènes d'induction.
\end{enumerate}
\end{corrige}

\begin{corrige}
	\begin{enumerate}
		\item Initialement, l'aiguille de la boussole ne ressent que le
		  champ magnétique terrestre. Elle est donc alignée avec ce dernier
		  (plus exactement avec la composante horizontale de ce dernier).
		  En revanche, lorsque les bobines de Helmholtz sont alimentées,
		  elles vont à leur tour générer un champ magnétique, 
		  non aligné avec l'aiguille, qui va induire 
		  un couple sur cette dernière et donc la faire tourner.
		\item La boussole est à l'équilibre lorsque les couples induits
		  par les deux champs magnétiques s'annulent. Cette annulation
		  est atteinte lorsque \fbox{$B_h \sin \phi = B_0 \cos \phi$}, où
		  $B_0$ est le champ magnétique généré par les bobines de Helmholtz
		  et ressenti par la boussole. On a alors
		  \begin{equation*}
			  \boxed{\tan \phi = \dfrac{\mu_0 I}{2 R B_h}.}
		  \end{equation*}
	  \item L'application numérique donne \fbox{
	    $B_h = \unit{23.5}{\micro \tesla}$.}
\end{enumerate}
\end{corrige}

\begin{corrige}
\begin{enumerate}
\item Le cristal se trouvant dans un champ uniforme $\vecb$, l'énergie magnétique
  d'un moment magnétique $m_+$ est donné par 
  \begin{equation*}
	  \boxed{E_+ = -\vecm_+ \cdot \vecb = -\mu_B B.}
  \end{equation*}

  \item $Z$ est \fbox{sans dimension}. $k_B T$ est homogène à une énergie donc $k_B$
    s'exprime en $\joule \usk \rp \kelvin$ et donc en \fbox{$\kilogram \usk \meter
    \squared \usk \second \rpsquare \usk \rp \kelvin$}.
  \item D'après la première question, on a \fbox{$E_+ = - \mu_B B$ et $E_- = 
    \mu_B B$.}
    Pour déterminer l'expression de $Z$, il faut écrire que 
    \begin{equation*}
	    N = N_+ + N_- \iff \boxed{Z = \dfrac{1}{N}\left[
		    \exp\left(\dfrac{\mu_B B}{k_B T}\right) + 
    \exp\left(\dfrac{-\mu_B B}{k_B T}\right) \right].}
    \end{equation*}
    Cette expression permet d'aboutir à  
    \begin{equation*}
	    \boxed{N_\pm = \dfrac{N}{\exp\left(\dfrac{\mu_B B}{k_B T}\right) + 
		    \exp\left(\dfrac{-\mu_B B}{k_B T}\right)} 
	    \exp\left(\dfrac{\pm \mu_0 B}{k_BT}\right).}
    \end{equation*}

    Par définition, le moment magnétique de l'échantillon est obtenu en sommant
    tous les moments magnétiques atomiques
    \begin{equation*}
	    M = N_+ \mu_B - N_- \mu_B = N \mu_B
		    \dfrac{\exp\left(\dfrac{\mu_B B}{k_B T}\right) - 
		    \exp\left(\dfrac{-\mu_B B}{k_B T}\right)}
		    {\exp\left(\dfrac{\mu_B B}{k_B T}\right) + 
		    \exp\left(\dfrac{-\mu_B B}{k_B T}\right)}
	      = \boxed{N \mu_B \tanh\left(\dfrac{\mu_B B_0}{k_B T}\right).}
    \end{equation*}

    \item L'énergie magnétique d'un moment magnétique est négligeable dès lors que 
      \begin{equation*}
	      \mu_B B_0 \ll k_B T \iff \boxed{B \ll \dfrac{k_B T}{mu_B}.}
      \end{equation*}
      À temperature ambiante, $T = \unit{300}{\kelvin}$, l'application numérique
      donne $B = \unit{450}{\tesla}$. Cette intensité de champ est impossible à obtenir
      expérimentalement. On peut donc raisonnablement \fbox{négliger l'énergie magnétique
      devant l'énergie thermique.}
      \item L'aimantation du cristal s'obtient en écrivant
      \begin{equation*}
	      M_c = \dfrac{M}{V} = \dfrac{N \mu_B}{V} 
	      \tanh\left(\dfrac{\mu_B B_0}{k_B T}\right).
      \end{equation*}
      Au premier ordre, l'aimantation du cristal vaut
      \begin{equation*}
	      \boxed{M_c = \dfrac{N \mu_B^2 B_0}{k_B T V}.}
      \end{equation*}
      Dans un paramagnétique, la susceptibilité étant très faible devant $1$, 
      on a $B_0 = \mu_0 H_0$ où $H_0$ est l'amplitude du vecteur excitation magnétique.
      La susceptibilité $\chi$ s'écrit donc
      \begin{equation*}
	      \boxed{\chi = M_c/H_0 = \dfrac{N \mu_0 \mu_B^2}{V k_B T}.}
      \end{equation*}
       On retrouve bien la loi de Curie avec \fbox{$C = \dfrac{N \mu_0 \mu_B^2}
	{V k_B}$.}

      \item La susceptibilité 
      d'un cristal composé d'une mole de moments atomiques s'écrit
      \begin{equation*}
	      \chi = \dfrac{\mathcal{N} \mu_0 \mu_B^2 \rho}{\mathcal{M} k_B T}.
      \end{equation*}
      L'application numérique donne \fbox{$\chi = 7.51 \times 10^{-4}$.}
\end{enumerate}
\end{corrige}
