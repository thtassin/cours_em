\begin{corr}{Étude macroscopique de l'aimantation}
	\correction{Estimation de la taille du noyau}
\begin{corrlist}
	\item À la surface, le champ magnétique terrestre est de l'ordre de
	  $\unit{50}{\micro \tesla}$. Le rayon de la Terre vaut approximativement
	  $\unit{6400}{\kilo \meter}$. On a donc
	  
	  \begin{equation*}
		  \boxed{ m = \dfrac{4 \pi B R_T^3}{\mu_0} \approx \unit{10^{23}}{\ampere
		  \usk \meter \squared}.} 
	  \end{equation*}

	 \item On fait ici l'hypothèse que tous les moments magnétiques atomiques
	   sont alignés et orientés dans le même sens. On a alors
	   \begin{equation*}
		   m = N\mu_B \iff \boxed{N = \dfrac{m}{\mu_B} = 10^{46}.}
	   \end{equation*}

	 \item Le noyau contient $N$ atomes. Pour obtenir le nombre de mole 
	   $n$ que cela représente, il suffit de diviser $N$ par le nombre d'
	   Avogadro $\mathcal{N}_A = \unit{6.022 \times 10^{23}}{\rp \mole}$.
	   On a alors $n = N/\mathcal{N}_A$. On peut alors remonter à la masse 
	   $m_N$ du noyau en utilisant sa masse molaire $m_N = nM$. Finalement,
	   le volume s'écrit
	   \begin{equation*}
		   V = \dfrac{m_N}{\rho} = \boxed{\dfrac{mM}{\mu_B \mathcal{N}_A \rho}
	   \approx \unit{1.2 \times 10^{17}}{\cubic \meter}.}
	   \end{equation*}

   \item On considère que le noyau de la Terre est sphérique. Son rayon $R_N$ 
     s'écrit donc
     \begin{equation*}
	     R_N = \left(\dfrac{3V}{4\pi}\right)^{1/3} \approx 
	     \boxed{\unit{300}{\kilo \meter}.}
     \end{equation*}
     En réalité le noyau interne a un rayon de $\unit{1200}{\kilo \meter}$. Nous 
     avons donc sous-estimé ce dernier. En effet, notre principal erreur a été 
     de considérer que les moments magnétiques atomiques étaient alignés
     alors que la température dans le noyau est très élevée !
  \item La température du noyau est supérieur à la température du fer qui est alors
    paramagnétique. Il n'y a donc aucune raison que les moments dipolaires atomiques
    produisent un champ macroscopique. Le champ magnétique produit par la
    Terre résulte de phénomènes d'induction.
\end{corrlist}


\correction{Composante horizontale du champ géomagnétique}
	\begin{corrlist}
		\item Initialement, l'aiguille de la boussole ne ressent que le
		  champ magnétique terrestre. Elle est donc alignée avec ce dernier
		  (plus exactement avec la composante horizontale de ce dernier).
		  En revanche, lorsque les bobines de Helmholtz sont alimentées,
		  elles vont à leur tour générer un champ magnétique, 
		  non aligné avec l'aiguille, qui va induire 
		  un couple sur cette dernière et donc la faire tourner.
		\item La boussole est à l'équilibre lorsque les couples induits
		  par les deux champs magnétiques s'annulent. Cette annulation
		  est atteinte lorsque \fbox{$B_h \sin \phi = B_0 \cos \phi$}, où
		  $B_0$ est le champ magnétique généré par les bobines de Helmholtz
		  et ressenti par la boussole. On a alors
		  \begin{equation*}
			  \boxed{\tan \phi = \dfrac{\mu_0 I}{2 R B_h}.}
		  \end{equation*}
	  \item L'application numérique donne \fbox{
	    $B_h = \unit{23.5}{\micro \tesla}$.}
\end{corrlist}


\correction{Suscpetibilité et loi de Curie}
\begin{corrlist}
\item Le cristal se trouvant dans un champ uniforme $\vecb$, l'énergie magnétique
  d'un moment magnétique $m_+$ est donné par 
  \begin{equation*}
	  \boxed{E_+ = -\vecm_+ \cdot \vecb = -\mu_B B.}
  \end{equation*}

  \item $Z$ est \fbox{sans dimension}. $k_B T$ est homogène à une énergie donc $k_B$
    s'exprime en $\joule \usk \rp \kelvin$ et donc en \fbox{$\kilogram \usk \meter
    \squared \usk \second \rpsquare \usk \rp \kelvin$}.
  \item D'après la première question, on a \fbox{$E_+ = - \mu_B B$ et $E_- = 
    \mu_B B$.}
    Pour déterminer l'expression de $Z$, il faut écrire que 
    \begin{equation*}
	    N = N_+ + N_- \iff \boxed{Z = \dfrac{1}{N}\left[
		    \exp\left(\dfrac{\mu_B B}{k_B T}\right) + 
    \exp\left(\dfrac{-\mu_B B}{k_B T}\right) \right].}
    \end{equation*}
    Cette expression permet d'aboutir à  
    \begin{equation*}
	    \boxed{N_\pm = \dfrac{N}{\exp\left(\dfrac{\mu_B B}{k_B T}\right) + 
		    \exp\left(\dfrac{-\mu_B B}{k_B T}\right)} 
	    \exp\left(\dfrac{\pm \mu_0 B}{k_BT}\right).}
    \end{equation*}

    Par définition, le moment magnétique de l'échantillon est obtenu en sommant
    tous les moments magnétiques atomiques
    \begin{equation*}
	    M = N_+ \mu_B - N_- \mu_B = N \mu_B
		    \dfrac{\exp\left(\dfrac{\mu_B B}{k_B T}\right) - 
		    \exp\left(\dfrac{-\mu_B B}{k_B T}\right)}
		    {\exp\left(\dfrac{\mu_B B}{k_B T}\right) + 
		    \exp\left(\dfrac{-\mu_B B}{k_B T}\right)}
	      = \boxed{N \mu_B \tanh\left(\dfrac{\mu_B B_0}{k_B T}\right).}
    \end{equation*}

    \item L'énergie magnétique d'un moment magnétique est négligeable dès lors que 
      \begin{equation*}
	      \mu_B B_0 \ll k_B T \iff \boxed{B \ll \dfrac{k_B T}{mu_B}.}
      \end{equation*}
      À temperature ambiante, $T = \unit{300}{\kelvin}$, l'application numérique
      donne $B = \unit{450}{\tesla}$. Cette intensité de champ est impossible à obtenir
      expérimentalement. On peut donc raisonnablement \fbox{négliger l'énergie magnétique
      devant l'énergie thermique.}
      \item L'aimantation du cristal s'obtient en écrivant
      \begin{equation*}
	      M_c = \dfrac{M}{V} = \dfrac{N \mu_B}{V} 
	      \tanh\left(\dfrac{\mu_B B_0}{k_B T}\right).
      \end{equation*}
      Au premier ordre, l'aimantation du cristal vaut
      \begin{equation*}
	      \boxed{M_c = \dfrac{N \mu_B^2 B_0}{k_B T V}.}
      \end{equation*}
      Dans un paramagnétique, la susceptibilité étant très faible devant $1$, 
      on a $B_0 = \mu_0 H_0$ où $H_0$ est l'amplitude du vecteur excitation magnétique.
      La susceptibilité $\chi$ s'écrit donc
      \begin{equation*}
	      \boxed{\chi = M_c/H_0 = \dfrac{N \mu_0 \mu_B^2}{V k_B T}.}
      \end{equation*}
       On retrouve bien la loi de Curie avec \fbox{$C = \dfrac{N \mu_0 \mu_B^2}
	{V k_B}$.}

      \item La susceptibilité 
      d'un cristal composé d'une mole de moments atomiques s'écrit
      \begin{equation*}
	      \chi = \dfrac{\mathcal{N} \mu_0 \mu_B^2 \rho}{\mathcal{M} k_B T}.
      \end{equation*}
      L'application numérique donne \fbox{$\chi = 7.51 \times 10^{-4}$.}
\end{corrlist}
\end{corr}
